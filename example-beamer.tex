\documentclass{beamer}

% Use the PSU theme
\usetheme{PSU}
\usepackage{psu-colors}
\usepackage{psu-logos}

\setmainfont{Acumin Pro}[
    UprightFont=*-Regular,
    BoldFont=*-Bold,
    ItalicFont=*-Italic,
]

% Title and author information
\title{Portland State University Beamer Template}
\subtitle{A Custom Beamer Theme}
\author{Your Name}
\institute{Portland State University}
\date{\today}

\begin{document}
	% Title page
    {
        \setbeamercolor{background canvas}{bg=psuForestGreen}
    \begin{frame}
		\titlepage
	\end{frame}
    }

    % Table of contents
    \begin{frame}
        \frametitle{Outline}
        \tableofcontents
    \end{frame}

	% Introduction slide
    \section{Introduction}
	\begin{frame}
		\frametitle{Introduction}
		\begin{itemize}
			\item This is a sample slide using the PSU Beamer theme.

			\item The theme incorporates PSU colors and logos.

			\item The font used is Acumin Pro.
		\end{itemize}
		% Insert the PSU logo at the bottom right corner
		% \psuLogo{color}{0.3}
	\end{frame}

	% PSU color palette slide: primary
    \section{PSU Color Palette}
    \subsection{Primary Colors}
	\begin{frame}
		\frametitle{PSU Color Palette}
		\showPSUColorsPrimary
	\end{frame}

	% PSU color palette slide: secondary
    \subsection{Secondary Colors}
	\begin{frame}
		\frametitle{PSU Color Palette}
		\showPSUColorsSecondary
	\end{frame}

	% Another slide
    \section{Content}
    \subsection{Block Elements}
	\begin{frame}
		\frametitle{Content Slide}
		\begin{block}{Block Title}
			This is a block of text within a content slide.
		\end{block}
        \begin{alertblock}{Alert Block Title}
            This is an alert block of text within a content slide.
        \end{alertblock}
	\end{frame}

    % Math slide
    \subsection{Mathematical environments}
    \begin{frame}
        \frametitle{Grothendieck Compactness Principle}
        \begin{alertblock}{Theorem}
            A norm closed subset $E$ of a Banach space $X$ is norm compact
            if and only if it is contained in the closed convex hull of a
            norm null sequence.
            I.e., $E \subset \overline{\text{conv}}(x_n)$ for some norm null
            sequence \(x_n \to 0\).
        \end{alertblock}
        Grothendieck himself attributes this result to J.~Dieudonné
        and L.~Schwartz.
    \end{frame}

    % Showcase other logos
    \subsection{Other Logos}
    \begin{frame}
        Some logos are provided in the \texttt{psu-logos} package: you have
        paths to the logos, and some direct wrappers...
        This is subject to change as the logos are not included in the 
        repository.
        \frametitle{Other Logos}
        \begin{figure}
            \centering
            \CLASLogoColor{1cm}
            \caption{College of Liberal Arts and Sciences Logo, in color,
            with height 1cm.}
        \end{figure}
        \begin{figure}
            \centering
            \includegraphics[width=0.3\textwidth]{\UHCPath/psu_uhc_4cp.jpg}
            \caption{University Honors College Logo, in color.}
        \end{figure}
    \end{frame}

	% Conclusion slide
    \section{Conclusion}
	\begin{frame}
		\frametitle{Conclusion}
		\begin{itemize}
			\item This concludes the sample presentation.

			\item Thank you for your attention!
		\end{itemize}
		% Insert the PSU logo
		% \psuLogo{color}{0.3}
	\end{frame}
\end{document}
