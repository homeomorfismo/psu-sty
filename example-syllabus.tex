\documentclass{article}
\usepackage{psu-syllabus}
\usepackage{psu-logos}
\usepackage{geometry}
\geometry{margin=1in}

\course{MTH 261 Introduction to Linear Algebra}
\department{Mathematics}
\sectionnumber{001}
\CRN{12345}
\schedule{MWF 10:00-10:50}
\instructor{John Doe}
\office{FMH 123}
\email{anemail@pdx.edu}
\officehours{MWF 11:00-12:00}
\website{https://www.pdx.edu}
\logo{\CLASPath/psu_clas.png}
\logoscale{0.5}

\begin{document}
    
    \maketitle

	\noindent
	Note that you can drop in without an appointment during my scheduled drop-in hours.
	These ``meetings'' will be one-on-one or small groups of students. You do not need
	to have a specific question or topic when coming to my drop-in hours. You also
	don't need to arrive right at the beginning or stay to the end.

	\section*{Course Description}
	Systems of linear equations, linear transformations, matrix algebra, vector spaces,
	and determinants.

	\noindent
	\textbf{Credits:} 4

	\noindent
	\textbf{Prerequisites:} MTH 251

	\section*{Learning Outcomes}
	At the end of this course, students will be able to:
	\begin{enumerate}
		\item Do matrix algebra.

		\item Solve systems of linear equations.

		\item Determine spanning sets and linear independent sets.

		\item Determine column, row, and null spaces.

		\item Understand vector spaces and linear transformations.

		\item Compute determinants.

		\item Find eigenvalues and eigenvectors.

		\item Improve problem solving skills and the relationship between algebraic and
			graphical representations of a function.
	\end{enumerate}

	\section*{Topics}
	\begin{enumerate}
		\item Systems of equations

		\item Row-reduction and echelon forms

		\item Matrix algebra

		\item Determinants

		\item Vectors spaces

		\item Vector algebra

		\item Spanning set

		\item Linear independence

		\item Column space

		\item Null space

		\item Linear transformations

		\item Real and complex eigenvalues, eigenvectors
	\end{enumerate}

	\section*{Required Materials}
	\textbf{Textbook -- W. Keith Nicholson, Linear Algebra with Applications:}

	\noindent
	This textbook is a free, open-source textbook available online at 
    \url{https://lyryx.com/linear-algebra-applications/}.

	\noindent
	\textbf{Lyryx OHM (Online Homework Manager) Account:}

	\noindent
	We will be using Lyryx for our weekly online homework. All materials for Lyryx
	OHM can be accessed fully through Canvas, this includes access to all on-line homework
	assignments and textbook. Before access is granted, however, each student must
	set up an account with Lyryx in Canvas. Your instructor will walk you through this
	procedure on the first day of class. To set up your Lyryx account in Canvas
	you can either:
	\begin{enumerate}
		\item Use access code purchased from Bookstore,

		\item Purchase an access code directly through the publisher, or

		\item Lyryx has temporary access codes available to students that require them
			(awaiting financial aid / credit card issues / not able to pay at that
			time). If you require any temporary pin codes, please feel free to let your
			instructor know.
	\end{enumerate}
	Once an access code is used, it is not refundable. Note that the costs may be different
	and financial aid may require you to use the bookstore to get reimbursed for the
	access code cost.

	\section*{Assessments}
	\begin{enumerate}
		\item \textbf{Quizzes: 20\% final grade.} 30 minutes long, at the end of the
			class.
			\begin{enumerate}
				\item (Week 3) Wednesday, January 22, 2025. Two first weeks of content.

				\item (Week 8) Wednesday, February 26, 2025. Week 4 to week 7.
			\end{enumerate}

		\item \textbf{Midterm: 25\% final grade.} (Week 6) Wednesday, February 12,
			2025. Week 1 to week 5.

		\item \textbf{Final exam: 35\% final grade.} (Week 11) Tuesday, March 18,
			2025. All the content. From 8:00 to 9:00.

		\item \textbf{Online homework: 20\% final grade.} (Weekly, starting week 2).
	\end{enumerate}

	\section*{Grading}
	The final grade is assigned as follows: A: 90\%+, B: 80\% - 89.99\%, C: 70\% -
	79.99\%, D: 60\% - 69.99\%, F: below 60\%. Your percentage will be calculated
	by dividing total points earned by total without extra credit (this should be done
	automatically in Canvas). The school's grading policy can be found at
    \url{https://www.pdx.edu/registration/grading-system}.

	\noindent
	All graded work will be evaluated on mathematical correctness, syntax (use of symbols
	and notation of mathematics), format (organization and presentation), the
	clarity of your arguments and reasonableness of your answers. Answers are not enough:
	justify all work by showing process and presenting a logical argument. Verify results
	whenever possible.

	\section*{Course Attendance and Make-up Policies}
	\begin{itemize}
		\item \textbf{Mandatory Attendance:} A minimum attendance rate of 75\% is required
			for successful course completion. Attendance below 50\% will result in automatic
			failure of the course, and the student will not be allowed to participate
			in the final examination. Students are obligated to inform the instructor of
			any absences in advance. Should attendance fall below 75\%, the instructor
			will engage with the student to determine an appropriate resolution.

		\item \textbf{Quizzes:} No make-up opportunities are available for missed
			quizzes.

		\item \textbf{Midterm Grade Recovery:} Students may recover up to 15\% of their
			midterm grade, provided they address all instructor feedback.

		\item \textbf{Assignment Extension Requests:} Requests for extensions on
			online assignments must be submitted no later than the conclusion of week
			7.

		\item \textbf{Extra Credit:} The instructor may offer extra credit
			assignments to the entire class, capped at 5\% of an individual assignment's
			grade.
	\end{itemize}

	\section*{Flexibility Statement}
	The instructor reserves the right to modify the course content and/or
	substitute assignments and learning activities in response to institutional, weather,
	or class situations.

	\section*{Course Calendar/Schedule}
	See above, assessments.

    \PSUPolicies
\end{document}
